\chapter{Limitations and Future Directions}\label{chap:discussion}

The primary (and obvious) limitation to this study is the narrow subject pool from which the data was gathered.  The data, being observational in nature and from a single athlete's history, allows us no ability to draw conclusions that accurately reflect a causal relationship between the predictors and the outcome variable. Additionaly, the absence of a random sampling mechanism would make any inferrening our conclusions to any population, other than Kobe Bryant himself, highly suspect.  Even with such a limited scope, the conclusions produced from our study serves an important purpose.  \par
%
Our conclusions, presented in the previous section, yielded strong evidence of Kobe's ability to make shots dwindling as his distance from the basket grew. The general principle behind this conclusion seems obvious, and it is.  Of course an athlete is going to see diminished shooting accuracy as they take longer shots.  What is significant about our results is not even the quantification of that relationship between accuracy and distance directly, but it is the proof that such a relationship is reliably quantifiable and, holding all other variables constant, interpretable in a 2-dimensional space.  These findings build a foundation on which future studies will be able to design experiments and investiagte broader populations. Future directions for our work may be investigating similar trends across NBA players of different vintages, skill levels, positions, and team compositions.  Such an experiment, with the appropriate design, would provide definitive insight into the extensibility of our findings in this analysis.



