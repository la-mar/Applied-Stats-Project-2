\chapter{Approach}\label{chap:approach}


\section{Problem Definition}
This study was birthed out of a need to reliably quantify three key metrics used in assessing a profession basketball player's shooting ability.  Those metris are:

1. Odds of making a shot as distance from the basket increases.

2. Linearity of the decline rate of the probability of making a shot with respect to the distance the shot was taken from the basket.

3. The relationship between the distance from the shooter to the basket and the odds of the shot being made is different when in the regular season verses the post-season.

Rephrasing those metrics as questions helps clarify the focus of the analysis as it relates to Kobe Bryant.
Those questions are as follows:

1. Do the odds of Kobe making a shot decrease with respect to distance he is from the hoop?

2. Does the probability of Kobe making a shot descrease linearly with respect to the distance he is from the hoop?

3. Is the relationship between the distance Kobe is from the basket and the odds of him making the shot different if they are in the playoffs.


To appropriately answer the questions of interest above, we must fit a series of classifier models to the training portion of our dataset, iteratively scoring and comparing the various models against one another so that we can tune their parameters and hone in on a final feature set.
Below, we set the stage for conducting our analysis through an exploratory analysis of the dataset.


\section{Exploratory Analysis}





% \section{Methodology}

\section{Evaluation}

\section{Illustration}
